\documentclass{article}
\usepackage{amsmath,amssymb}
\usepackage{tikz}
\usepackage[makeroom]{cancel}
\title{Coordinate Transformations}
\author{Blake Hegerle}

\renewcommand{\vec}[1]{\boldsymbol{#1}}

\newenvironment{xypic}[1]{
\begin{center}
    \begin{tikzpicture}
        % \draw[thin,lightgray!40] (-#1,-#1) grid (#1,#1);
        \draw[<->] (-#1,0)--(#1,0) node[right]{$x$};
        \draw[<->] (0,-#1)--(0,#1) node[above]{$y$};
}{
    \end{tikzpicture}
\end{center}
}

\begin{document}
\section*{Vector ``Multiplication''}

As you've seen, vectors can be added and subtracted. What about multiplication?

Vectors are completely made up, so you could just continue and \textbf{make 
up} a multiplication for vectors, with the intention is that it will work
like as much like regular multiplication as possible. 
This means we should take multiplication on the number line as a source of 
inspiration for how multiplication of vectors should work.

A vector has magnitude and direction. Magnitude for regular numbers is very easy:
just take the absolute value. For direction, consider positive numbers as
``pointing'' to the right, and negative to the left, so like a vector,
a number has magnitude and direction.
\begin{align*}
    +3 && \text{magnitude} &= 3\\
    -3 && \text{magnitude} &= 3\\
    0 && \text{magnitude} &= 0\\
    x && \text{magnitude} &=|x|
\end{align*}
We are going to keep up this pattern, and write
\[
    |\vec{v}| = \text{length of vector} = \sqrt{x^2+y^2}
\]

Back to the number line:
\begin{align*}
     2\times 3 &= 6 & \text{right} \times \text{right} &= \text{positive}\\
     2\times -3 &= -6 & \text{right} \times \text{left} &= \text{negative}\\
     -2\times 3 &= -6 & \text{left} \times \text{right} &= \text{negative}\\
     -2\times -3 &= 6 & \text{left} \times \text{left} &= \text{positive}
\end{align*}
What is the pattern? The magnitude of the result is always $\pm 6$. If the 
directions are the same,
like the first and last rows, then the result is positive. 
Otherwise, if the directions are opposite, the result is negative.
\begin{equation}
    \label{mult-ex-1}
    \pm 2 \times \pm 3 = 2 \times 3 \times 
    \begin{cases}
        1 & \text{same direction} \\
        -1 & \text{opposite direction}
    \end{cases}
\end{equation}

So vector multiplication should follow an analogous rule. 
Geometrically, two vectors pointing in the ``same direction'' means
$0^\circ$ between the two vectors. 
So think about the alignment of these vectors
\begin{xypic}{2}
    \draw[-stealth] (0,0)--(1.414216,1.414216) 
        node[anchor=north west]{$\vec{u}$ length 2};
    \draw[-stealth] (0,0)--(1.732051,1.732051) 
        node[anchor=south west]{$\vec{v}$ length 3};
\end{xypic}
Since $\vec u$ and $\vec v$ point in exactly the same direction, 
there are $0^\circ$ degrees between them.
\begin{xypic}{2}
    \draw[-stealth] (0,0)--(1.414216,1.414216) 
        node[anchor=south west]{$\vec{u}$ length 2};
    \draw[-stealth] (0,0)--(-1.732051,-1.732051) 
        node[anchor=north east]{$\vec{w}$ length 3};
\end{xypic}
Since $\vec u$ and $\vec w$ point in exactly opposite directions, 
there are $180^\circ$ degrees between them.

So you want function which is:
\begin{itemize}
    \item nice and smooth
    \item equal to $+1$ at $0^\circ$, and also
    \item equal to $-1$ at $180^\circ$.
\end{itemize}
There are an infinite number of functions which fit those requirements, 
and it is interesting to think about the implications of each choice,
but let's just stick with the nice, familiar choice of cosine.

Look back at Eq.~\ref{mult-ex-1}, and we're very close to a definition
for vector multiplication. 
\[
    \vec{u} \cdot \vec{v} 
    = \text{magnitude of } \vec u \times \text{magnitude of } \vec v \times \text{function of angle}
\]
Naming the angle between $\vec u$ and $\vec v$ to be $\theta$, this is what we've
been working toward:
\begin{equation}
    \label{mult-ex-2}
    \vec{u} \cdot \vec{v} 
    = |\vec u| |\vec v| \cos \theta
\end{equation}

We'll need to know the cosine of the angle between a vector and the x axis:
\begin{xypic}{2}
    \draw[-stealth] (0,0)--(1, 1) 
        node[anchor=south east]{$\vec{v}$};
    \draw[-stealth] (0,0)--(1, 0) 
        node[anchor=north west]{$v_x$};
    \draw[-stealth] (1,0)--(1, 1) 
        node[anchor=north west]{$v_y$};
\end{xypic}
From this diagram:
\[
    \cos\theta = \frac{v_x}{\sqrt{v_x^2+v_y^2}}
\]

Now a little algebra. Denote the angle between $\vec u$ and the x axis
as $\theta_u$, likewise $\theta_v$. 
\begin{align*}
    \theta &= \theta_u - \theta_v\\
    |\vec u| |\vec v| \cos \theta 
    &= |\vec u| |\vec v| \cos (\theta_u - \theta_v)\\
    &= |\vec u| |\vec v| (\cos\theta_u\cos\theta_v + \sin\theta_u\sin\theta_v)\\
    &= |\vec u| |\vec v| \left(\frac{u_x}{\sqrt{u_x^2+u_y^2}}\cos\theta_v 
        + \frac{u_y}{\sqrt{u_x^2+u_y^2}}\sin\theta_v\right)\\
    &= \sqrt{u_x^2+u_y^2} |\vec v| \left(\frac{u_x}{\sqrt{u_x^2+u_y^2}}\cos\theta_v 
        + \frac{u_y}{\sqrt{u_x^2+u_y^2}}\sin\theta_v\right)\\
    &= |\vec v| \left(u_x\cos\theta_v 
        + u_y\sin\theta_v\right)\\
    &= u_x v_x + u_y v_y
\end{align*}
This is the dot product.

\end{document}