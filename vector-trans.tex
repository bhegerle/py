\documentclass{article}
\usepackage{amsmath}
\usepackage{tikz}
\usepackage[makeroom]{cancel}
\title{Coordinate Transformations}
\author{Blake Hegerle}

\renewcommand{\vec}[1]{\boldsymbol{#1}}

\newenvironment{xypic}[1]{
        \begin{center}
    \begin{tikzpicture}
        % \draw[thin,lightgray!40] (-#1,-#1) grid (#1,#1);
        \draw[<->] (-#1,0)--(#1,0) node[right]{$x$};
        \draw[<->] (0,-#1)--(0,#1) node[above]{$y$};
}{
    \end{tikzpicture}
\end{center}
}

\begin{document}
\section*{Introduction}

A vector is a point in space. 

It is usually drawn as an arrow
from the origin $(0,0)$ to a point $(x,y)$. 

\textbf{Example}: $\vec v = (1,1)$:

\begin{xypic}{2}
    \draw[-stealth] (0,0)--(1,1) node[anchor=south west]{$\vec{v}$};
\end{xypic}

Vectors always start with their tail at the origin $(0,0)$.

The coordinate values depend on where you set the origin

Can use field coordinates: the origin is in the middle of the field.

Can alternately use robot-centric coordinate: the origin is in the middle of the 
robot.

You can go back and forth from field to robot coordinates. 

\textbf{Example}: The robot is at field vector $\vec r=(1,2)$ and sees a piece 
at $\vec p=(1.5,-0.5)$. \emph{The 
second vector is in robot coordinates.} What are
its field coordinates? It is at $(2.5,1.5)$:
\begin{xypic}{3}
    \draw[-stealth] (0,0)--(1,2) node[anchor=south east]{$\vec{r}$};
    \draw[-stealth] (1,2)--(2.5,1.5) node[anchor=south]{$\vec p$};
    \draw[-stealth] (0,0)--(2.5,1.5) node[anchor=north west]{};
\end{xypic}
Add the $x$ and $y$ values separately: 
\begin{align*}
    \vec r+\vec p &=(1,2)+(1.5,-0.5) 
    &= (1+1.5,2-0.5) \\
    &= (2.5,1.5)
\end{align*}
    
You can go the other way, too.

\textbf{Example}: A piece is at field vector $\vec q=(2,-1)$.
What are its robot coordinates? 

Subtract:\begin{align*}
    \vec q-\vec r &=(2,-1)-(1,2)\\
                  &=(1,-3)
\end{align*} 
\begin{xypic}{3}
    \draw[<->, dotted] (-3,2)--(3,2) node[right]{$x_r$};
    \draw[dotted,<->] (1,-3)--(1,3) node[above]{$y_r$};
    \draw[-stealth] (0,0)--(1,2) node[anchor=south east]{$\vec{r}$};
    \draw[-stealth] (1,2)--(2,-1) node[anchor=south]{};
    \draw[-stealth] (0,0)--(2,-1) node[anchor=north west]{$\vec q$};
\end{xypic}


Think about there being different $xy$ axes at centered on the robot in the previous
drawing.
The vector $\vec p$ is always the same physical point. The coordinates are all that
changes.

\section*{Rotation}

What if the robot is turned $-30^\circ$ clockwise? What will be the coordinates of a vector
$\vec p=(2, 0.5)$ after the rotation?

In robot coordinates, the piece will appear to rotate $+30^\circ$ (counterclockwise).

\begin{xypic}{3}
    \draw[-stealth] (0,0)--(2, 0.5) node[anchor=east]{$\vec p$};
    \draw[-stealth] (0,0)--(1.482050808, 1.433012702) node[anchor=east]{$\vec p$};
    \draw[-stealth,dotted] (2, 0.5) arc (14.03624347:44.03624347:2.061552813) 
                            node[anchor=west]{$30^\circ$};
\end{xypic}

This is pretty to easy to figure out if you think about $\vec p$ in polar coordinates.

Before rotating, $\vec p$ is at polar coordinates $(r,\theta)$. 
After, the new coordinates will be
$(r,\theta+30^\circ)$.
Again, the coordinates changed even though the vector did not move physically.

Now, the basic polar coordinate equations are
\begin{align*}
    r&=\sqrt{x^2+y^2}\\
    x&=r\cos\theta\\
    y&=r\sin\theta
\end{align*}
Now for a little algebra; remember your angle addition formula?
\begin{align*}
    x_r &= r \cos(\theta+30^\circ)\\
        &= r(\cos\theta\cos 30^\circ - \sin\theta\sin 30^\circ)\\
    \cos\theta &= \frac xr\\
    \sin\theta &= \frac yr\\
    x_r &= {\cancel r}\left(\frac x{\cancel r}\cos 30^\circ - \frac y{\cancel r}\sin 30^\circ\right)\\
        &= x\cos 30^\circ - y\sin 30^\circ
\end{align*}

In exactly the same way, we can derive
\begin{align*}
    y_r &= r\sin(\theta+30^\circ)\\
        &= r(\sin\theta\cos 30^\circ + \cos\theta\sin 30^\circ)\\
    y_r &= {\cancel r}\left(\frac y{\cancel r}\cos 30^\circ + \frac x{\cancel r}\sin 30^\circ\right)\\
        &= x\sin 30^\circ + y\cos 30^\circ 
\end{align*}

Make a note of the coefficients in the final equations:
\[
    \begin{bmatrix}
        \cos 30^\circ & -\sin 30^\circ\\
        \sin 30^\circ & \cos 30^\circ
    \end{bmatrix}
\]

You'll see that again.

You can use these two techniques to do coordinate transformations. 
As the robot moves left and right, you just need to add or subtract vectors.
As the robot rotates, you need to use the rotation equations.
\end{document}